\documentclass{article}
\usepackage{graphicx} % Required for inserting images

\title{FRANCE CLASSES}
\author{Ewerthon Araujo Melzani}
\date{2023}

\begin{document}

\maketitle

\section*{Classe 2}

\vspace*{1.5 cm}


Salut Antoine, comment ça va?

Je viens de rentrer de vacances et j'ouvre mon courrier pour la primière fois depuis le mois d'août. Quelle surpreise et quelle
bonne nouvelle ! Tu viens d'être papa, c'est génial! C'est un garçon, tu dois être ravi n'est-ce pas? Mais comment s'appelle-t-il? 
Tu ne le dis pas dans ton mail! En tout cas, il est très mignon sur la photo! Pourrais-tu nous envoyer d'autres photos? Crois-tu qu'on
pourra se voir bientôt pour fêter ce grand événement ?

Depuis Paris, avec Julie, on vous envoie, à toi et bien sûr à Marie, toutes nos félicitations.


\vspace*{1.5 cm}

\begin{enumerate}
    \item Je dis/ Tu dit/ Il dit
    \item Nous disons/ Vous dites/ Ils disent
\end{enumerate}

\vspace*{1.5 cm}

\begin{enumerate}
    \item Vous allez souvent au cinéma?
    \item Qui est cet homme? c'est un ami de Julie.
    \item En France, Il y a soixante milions d'habitants.
    \item Le soir, je me promène dans la rue.
    \item C'est l'hiver et il fait froid.
    \item Vous vous levez tôt le matin? -Oui, je me lêve très tôt.
    \item Qu'est-ce que vous buvez/prenez au petit déjeuner? Du café.
    \item Quelle est votre profession? Je suis médecin.
    \item Tous les hommes doivent manger pour vivre.
    \item J'habite depuis vingt ans dans le même quartier.
    \item Bientôt, il va faire chaud. C'est l'été.
    \item Hier, les enfants ont regardé la télévision pendant deux heures.
    \item Excusez-moi, je suis pressé: je vais/dois partir!
    \item Quel est le bus qui va a l'aéroport?
    \item Tu vas au travail à pied? Non, j'y vais en voiture!
    \item Nous sommes rentrés de vacances hier.
    \item Quand je suis arrivé à Paris, il faisait froid.
    \item Voilà la liste des livres dont j'ai besoin pour mes études.
    \item Dis-mois ce que tu penses de ma nouvelle coiffure.
    \item Quel est le jour où vous êtes né?
    \item Quand je serai en Grèce, je vous écrirai une carte.
    \item Marie aime les fraises et elle les mange tous les jours.
    \item David est une personne que nous aimons beaucoup.
    \item Hier soir, nous avons vu un beau film à la téle.
    \item Nous allons au cinéma, tu viens avec nous?
    \item La Tour Eiffel a été construite il y a plus de 100 ans.
    \item Il y a quelqu'un dans la salle 1319? Non, il n'y a personne.
    \item Qu'est-ce qui se passe dehors? Je ne sais pas ce qui se passe?
    \item Mozart a composé sa primière ouvre quand il avait six ans.
    \item Jean est à l'hôpital, il a eu un accident de voiture hier soir.
\end{enumerate}

\vspace*{1.5 cm}

\subsubsection*{Conjuger les verbes au passé composé avec l'auxiliare être:}

\begin{enumerate}
    \item elle s'est souvenue
    \item sont rentrés
    \item se sont disputées
    \item sont parties
    \item sommes nés/nées
    \item es descendu.e
    \item me suis promené
    \item est sortie
    \item êtes allé.e.s
    \item sont devenus
\end{enumerate}

\subsubsection*{Conjuger les verbes au passé composé:}

\begin{enumerate}
    \item J'ai oublié / m'a coupé
    \item ont eu / ont passé
    \item ont arrêté / ont arrivés
    \item a prévenu
    \item a monté
    \item se sont trompés / ont eu
    \item avons pris / sommes alleś
    \item a adoré / est passée
    \item vous êtes levés / n'ai pas entendu
    \item ont tombés / se sont mariés
\end{enumerate}

\subsubsection*{Répondre aux questions:}

\begin{enumerate}
    \item Non, je ne l'ai pas envoyé / Je ne le lui ai pas envoyé
    \item Oui, nous y sommes allés/ On y est allés
    \item Oui, elles ont vu la nouvelle maison/ Elles l'ont vue 
    \item Non, il n'est pas veny en bus
    \item Non, je ne l'ai pas compris
    \item Oui, je l'ai déjá mise/ Oui, nous l'avons déjà mise
    \item Non, je ne l'ai pas appelée
    \item Oui, Vous nous l'avez déjà présenté
    \item Non, je ne l'ai pas faite  
    \item Non, je n'y ai pas répondu/ Nous n'avons pas répondu à toutes
\end{enumerate}

\vspace*{1.5 cm}


\subsubsection*{Questions:}

\begin{enumerate}
    \item D'une formation universitaire, plus personnel et encadrés a chaque de personne
    \item Étudie avec son propre ordinateur 
    \item Les jeunes que habitent dans des villes éloignées 
    \item Alex - BDS Gestion Forestière / Mathis - STAPS / Morgane Sciences de l'Éducation
    \item Plus proche de ton propre maison , un bel endroit , mieux d'un point de vue financiar et plus facile de concentrer parce que il a moins de eleves
\end{enumerate}


\LARGE{Ewerthon Araujo Melzani}

\vspace*{1.5 cm}

\huge{!SO O GAGA SALVA!}

\end{document}
